\chapter{Commutative algebra of numerical monoid rings}

All our rings are assumed to be commutative and unital. The symbol $\kk$ will always denote a field and $\kk^*=\kk\setminus\{0\}$. Also, all our monoids are commutative and ring homomorphisms are assumed to respect the units.

\section{Algebras}

Let $\kk$ be a field and $A$ a ring. Then, $A$ is called a \emph{$\kk$-algebra} if $A$ contains a isomorphic copy of $\kk$ as a subring. For simplicity of notation, we will identify $\kk$ with its isomorphic copy in $A$. Every $\kk$-algebra is also a $\kk$-vector space. For two $\kk$ algebras $A$ and $B$, a ring homomorphism $f:A\to B$ is a \emph{$\kk$-algebra homomorphism} if it is also a $\kk$-linear map.

The basic example of a $\kk$-algebra is the multivariate polynomial ring $\kk[X_1,\ldots,X_n]$. The quotient of a $\kk$-algebra by an ideal is also a $\kk$-algebra.

A $\kk$-algebra $A$ is called \emph{finitely generated} if there is a finite family of elements $\{a_1,\ldots,a_n\}\subset A$, called \emph{generators of $A$}, such that $A$ is the smallest sub-algebra of $A$, containing the $a_i$'s. In this case we will write $A=\kk[a_1,\ldots,a_n]$. The assignment $X_i\mapsto a_i$, $i=1,\ldots n$, gives rise to a surjective $\kk$-algebra homomorphism
$$
\kk[X_1,\ldots,X_n]\to A.
$$
Hence, the \emph{Isomorphism Theorem} for rings implies that \emph{every} finitely generated $\kk$-algebra is isomorphic to a quotient of the form $\kk[X_1,\ldots,X_n]/I$ for some natural number $n\in\NN$ and an ideal $I\subset\kk[X_1,\ldots,X_n]$.

Notice, a (finite) generating set of a $\kk$-algebra $\kk[a_1,\ldots,a_n]$ is highly non-unique, not even if the $a_i$ are variables and we only consider generating sets of the smallest size. In fact, we have
\begin{align*}
\kk[X_1&,\ldots,X_n]=\kk[\mu_1X_1+\lambda_1,\ldots,\mu_nX_n+\lambda_n]=\\
&\kk[X_1,X_2+F_2(X_1),X_3+F_3(X_1,X_2),\ldots,X_n+F_n(X_1,\ldots,X_{n-1})],
\end{align*}  
for arbitrary elements
\begin{enumerate}[label=(\alph*)]
\item $\mu_i\in\kk^*$ and $\nu_i\in\kk$, where $i=1,\ldots,n$,
\item $F_i(X_1,\ldots,X_{i-1})\in\kk[X_1,\ldots, X_{i-1}]$, where $i=2,\ldots,n$
\end{enumerate}

\section{Monoid rings}

Let $M$ be a monoid. The monoid algebra $\kk[M]$ is the $\kk$-vector space over the basis $M$, where the multiplicative structure is defined by
\begin{align*}
\left(\sum_i\lambda_im_i\right)
\cdot\left(\sum_j\mu_jn_j\right)=\sum_{i,j}(\lambda_i\mu_j)(m_in_j),
\end{align*} 
where $\lambda_i,\mu_j\in\kk$, $m_i,n_j\in M$, and the monoid operation is written multiplicatively. The reader is referred to \cite[Chapter 2]{Kripo} for generalities on monoid rings. 

The monoid ring $\kk[M]$ is a $\kk$-algebra, where $\kk$ embeds to $\kk[M]$ via $\lambda\mapsto\lambda\cdot 1$ for the neutral element $1\in M$ (changed from the additive notation $0\in M$). 

The algebra $\kk[M]$ also contains an isomorphic copy of $M$ via the embedding $m\mapsto 1\cdot m$, where $1\in\kk$.

Notice, $1\in\kk$ and $1\in M$ are the same unit element of $\kk[M]$.

We will write elements of $\kk[M]$ as linear combinations $\sum_i\lambda_im_i$, with the understanding that $1\in M$ gets identified with $1\in\kk$ (and the monoid operation is written multiplicatively).

The defining universal property of the monoid ring $\kk[M]$ is that, for any $\kk$-algebra $A$ and any monoid homomorphism $f:M\to A$ with respect to the multiplicative structure of $A$, there exists a unique $\kk$-algebra homomorphism $\kk[M]\to A$, extending $f$.  Moreover, this universal property defines the algebra $\kk[M]$ uniquely, up to isomorphism.

\begin{example}\label{polynomial_example}
The basic examples of a monoid ring is the polynomial rings $\kk[(\ZZ_+)^n]=\kk[X_1,\ldots,X_n]$, where the identification of the two $\kk$-algebras is through the $\kk$-algebra isomorphism, defined by 
$$
(a_1,\ldots,a_n)\mapsto X_1^{a_1}\cdots X_n^{a_n}.
$$
\end{example}

\begin{example}\label{numerical_polynomial}
For a numerical monoid monoid $M$, the monoid algebra $\kk[M]$ can be though of as the subalgebra of the univariate polynomial ring $\kk[X]$, consisting of the polynomials, whose reduced forms only involve monomials of the form $\lambda X^m$, where $\lambda\in\kk$ and $m\in M$.
\end{example}

\section{Embedding dimension and multiplicity}\label{terminology}
Assume $M\subset\ZZ_+$ is a numerical monoid, generated by coprime numbers $m_1,\ldots,m_e\in\NN$. Then we have the $\kk$-algebra homomorphisms:

\begin{align*}
&f:\kk[X_1,\ldots,X_e]\to\kk[M],\quad X_i\mapsto m_i,\quad i=1,\ldots,e,\\
&g:\kk[X]\to\kk[M],\quad X\mapsto m_1.
\end{align*}
For the corresponding affine varieties (assuming $\kk$ is algebraically closed), via \emph{Hilbert Nullstellensatz} \cite[Chapter 7]{Atiyah}, $f$ induces an algebraic embedding of the monomial curve $\spec(\kk[M])$ into the affine space $\mathbb A^n_\kk$. This monomial curve is given by 
\[
	\{(t^{m_1}, t^{m_2}, \ldots, t^{m_e}) \mid  t \in \kk \}
\]
 and an algebraic surjection from the $\spec(\kk[M])$ to the affine line $\mathbb A_\kk^1$, given by 
\[
	(t^{m_1}, t^{m_2}, \ldots, t^{m_e}) \longmapsto t^{m_1}, \ \ \ t \in \kk,
\]
which is generically of the form `$n$ points to one point\rq{}. Correspondingly, $e$ is called the \emph{embedding dimension} of $M$ with respect to the given generators and $m_1$ is called the \emph{multiplicity} of $M$. Notice, according to Lemma \ref{system of generators}, the multiplicity is independent of the choice of generators. This terminology also explains our use of $e$ for the number of generators of $M$.

\section{Graded rings}
A \emph{graded} $\kk$-algebra $A$ is an algebra, admitting a \emph{direct sum} representation
\begin{align*}
A=A_0\oplus A_1\oplus A_2\oplus\cdots
\end{align*}
where $A_i\subset A$ is a $\kk$-vector subspace, respecting the multiplicative structure as follows: $A_i\cdot A_i\subset A_{i+j}$ for every $i,j\in\ZZ_+$.

We call $A_n$ the \emph{degree $n$-homogeneous component} of $A$. For an element $a\in A_n\setminus\{0\}$, we write $\deg(a)=n$.

A $\kk$-algebra can carry several different graded structures. For instance, we can make the polynomial ring $\kk[X_1,\ldots,X_n]$ into a graded algebra in infinitely many different ways.
\begin{example}\label{graded_algebra}
Every family of natural numbers $c_1,\ldots c_n\in\NN$ defines a graded structure
\begin{align*}
\kk[X_1,\ldots,X_n]=\kk\oplus A_1\oplus A_2\oplus\cdots,
\end{align*}
via putting $\deg(X_i)=c_i$. In more detail, under this grading we have
\begin{align*}
A_i=\left\{\sum_t\lambda_tX_1^{a_{t1}}\cdots X_n^{a_{tn}}\quad \bigg|\quad \lambda_t \in \kk, \ c_1a_{t1}+\cdots+c_na_{tn}=i\right\}.
\end{align*}
When $c_1=c_2=\cdots=c_n=1$, we get the \emph{standard} grading of the polynomial ring.
\end{example}

Let $A=A_1\oplus A_2\oplus\cdots$ be a graded $\kk$-algebra, such that the $\kk$-vector space dimension $\dim_\kk(A_n)$ is finite for every $n$. Then the function $H_A:\ZZ_+\to\ZZ_+$, defined by $H_A(n)=\dim_\kk(A_n)$ is the \emph{Hilbert function} of $A$ (for this grading).

\begin{example}
For the standard grading of $A=\kk[X_1,\ldots,X_n]$, we have 
\begin{align*}
H_A(i) = \binom{n + i - 1}{i} \qquad \text{(see \cite[p. 45]{Eisenbud})}.
\end{align*}
\end{example}

We will need the following consequence of the general dimension theory of graded rings \cite[Chapter 11]{Atiyah}, where $\sqrt{0}$ denotes the ideal of all nilpotent elements -- the \emph{nil-radical:}

\begin{lemma}\label{dimension1}
Let $A=\kk\oplus A_1 \oplus A_2\oplus\cdots$ be a finitely generated graded $\kk$-algebra such that $A/\sqrt{0}\cong\kk[X]$ as $\kk$-algebras, then $A$ is a one-dimensional ring and, consequently, $H_A$ is eventually a constant function, i.e., $H_A(i)=H_A(i+1)$ for $i > 0$.
\end{lemma}

\begin{remark}\label{why_interesting}
We point out that dimension theory in commutative algebra does \emph{not} say how large $i$ needs to be to guarantee the equality $H_A(i)=H_A(i+1)$.
\end{remark}

For $\kk$-algebra $A$ and an ideal $I\subset A$, one defines the \emph{associated graded algebra} as follows:
\begin{align*}
\gr_I(A)=A/I\oplus I/I^2\oplus I^2/I^3\oplus\cdots,
\end{align*}  
where:
\begin{enumerate}[label=(\alph*)]
\item
 $I^k$ is the \emph{$k$-th power of $I$,} i.e. the ideal generated by all possible $k$-fold products $a_1\cdots a_k$, where $a_1,\ldots,a_k\in I$,
\item The multiplicative structure is determined by the pairings:
\begin{align*}
I^k/I^{k+1}\times I^l/I^{l+1}&\to I^{k+l}/I^{k+l+1},\\
(\overline{a},\overline{b})&\mapsto\overline{ab},\\
&\qquad\qquad k,l\in\ZZ_+.
\end{align*}
(Here we assume $I^0=A$.)

The importance of this construction is that the affine scheme of $\gr_I(A)$ is a \emph{flat deformation} of that of $A$, called the \emph{Rees deformation} \cite[Section 6.4]{Eisenbud} and it plays an important role in resolutions of singularities in algebraic geometry.
\end{enumerate}

