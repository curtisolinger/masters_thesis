\chapter{Numerical monoids and the Frobenius number}

\section{Numerical monoids}

In this paper we are only interested in the submonoids of the nonnegative integers. These submonoids give a rich source of material to study. We therefore come to an important definition.

\begin{definition}\label{numerical monoid}
A \emph{numerical monoid} is a submonoid of $\Z_+$ with finite complement in $\Z_+$.
\end{definition}

The set $\{0, 6, 7, 8, 9, \rightarrow \}$ is a numerical monoid, whereas the set of even nonnegative integers $2 \Z_+$ is not because its complement in $\Z_+$ is the infinite set of odd integers. 

Since the complement of a numerical monoid $M$ in $\Z_+$ is finite, there must be a $m_0 \in M$, such that for all $m > m_0$, $m + 1 \in M$. In other words, the sequence of numerical monoid elements includes all consecutive integers after some point. In this light, the following is a proposition that can be used as the definition of a numerical monoid. 

\begin{proposition}\label{gcd}
Let $M$ be a submonoid of $\Z_+$. Let $G$ be the subgroup of $\Z$ generated by $M$. Then $M$ is a numerical monoid if and only if $1 \in G$, i.e., $G = \Z$.
\end{proposition}

\begin{proof}
Let $M$ be a numerical monoid. Since the complement of $M$ in $\Z_+$ has finite cardinality, there exists $m \in M$ such that $m + 1 \in M$. Thus for $m, m + 1 \in G$, we have $(m + 1) - m = 1 \in G$.

Now assume $1 \in G$ which implies there exist $m \in M$ such that $(m + 1) - m = 1$. Thus $m + 1 \in M$. Now we have to show $M$ has a finite complement in $\Z_+$. We claim that if $n \geq (m - 1)(m + 1)$, then $n \in M$. 

Let $n \geq (m - 1)(m + 1)$ and by the division algorithm there exists unique $q, r \in \Z$ such that $n = qm + r$, where $0 \leq r < m$. Notice that $(m - 1)(m + 1)$ = $(m - 1)m + (m - 1)$. So $n = qm + r \geq (m - 1)m + (m - 1)$. Then $q \geq m - 1$ and $m > r$ implies that $m - 1 \geq r$. Thus $q > r$. If we write, $n = qm + r = (q - r)m + r(m + 1)$, then $q - r > 0$ and $r > 0$. Therefore we have written $n$ as a nonnegative integer combination of $m$ and $m + 1$ which means that $n \in M$. Furthermore, any $n \geq (m - 1)(m + 1) \in M$. Thus the complement of $M$ in $\Z_+$ must be finite.
\end{proof}

Now we will focus on numerical monoids, given by generators. 

\begin{proposition}\label{gcd lemma}
Let $A$ be a nonempty subset of $\Z_+$. Then $\langle A \rangle$ is a numerical monoid if and only if $\gcd(A) = 1$.
\end{proposition}

\begin{proof}
The subgroup of $\ZZ$, generated by $\langle A \rangle$ is the  same as the subgroup generated by $A$. According to Proposition \ref{gcd}, the latter is $\Z$ if and only if $\langle A \rangle$ is a numerical monoid which, in turn, is equivalent to $\gcd(A) = 1$.
\end{proof}


\begin{proposition}
Let $M$ be a nontrivial submonoid of $\Z_+$. Then $M$ is isomorphic to a numerical monoid.
\end{proposition}

\begin{proof}
Let $d = \gcd(M)$ and let $S = \{ \frac{m}{d} : m \in M\}$. Thus the $\gcd(S) = 1$ and by Proposition \ref{gcd lemma}, $S$ is a numerical monoid and the map $\phi: M \rightarrow S$ by $\phi(m) = \frac{m}{d}$ is an isomorphism.
\end{proof}

For example $2\Z_+$ is a submonoid of $\Z_+$. All elements have the form $2x$ for some $x \in \Z_+$. Thus for $m \in 2\Z_+$, $\phi(m) = \phi(2x) = \frac{2x}{x} = x$. Therefore $2\Z_+ \cong \Z_+$.

Now some terminology about what it means to add two sets together. Let $A, B \subset \Z_+$. Then 
\[
  A + B = \{a + b: a \in A \text{ and } b \in B\}.
\]
Let $M$ be a numerical monoid and define $M^* = M \setminus \{0\}$. Then the set $M^* + M^*$ is the set of elements in $M$ that are the sum of two nonzero elements in $M$. For example if $M = \langle 2,3 \rangle$, then
\[
  M^* + M^* = \{4, 5, 6, 7, \ldots \}.
\]
Notice that the generators of $M$ are absent from this set.

\begin{lemma}\label{system of generators}
Let $M$ be a submonoid of $\Z_+$. Then $M^* \setminus (M^* + M^*)$ is a system of generators of $M$. Moreover, every system of generators of $M$ contains $M^* \setminus (M^* + M^*)$.
\end{lemma}

\begin{proof}
Let $m \in M^*$. We want to show that there exist $m_1, \ldots, m_k \in M^*~\setminus~(M^* + M^*)$ such that $m = a_1m_1 + \cdots + a_km_k$ for some $a_i \in \Z_+$. If $m \notin M^* \setminus (M^* + M^*)$ then there exist $x, y \in M^*$ such that $m = x + y$. Repeat this procedure with $x$ and $y$ and so on. The process must terminate because of the Well Ordering Principle: after each step we get smaller summands. This shows that $M^* \setminus (M^* + M^*) \not = \emptyset$ and, moreover, $M^* \setminus (M^* + M^*)$ generates $M$.

That every generating set of $M$ must contain $M^* \setminus (M^* + M^*)$ is straightforward \cite[Chapter 2.A]{Kripo}. 
\end{proof}

\begin{definition}\label{Hilbert_basis}
For a numerical monoid $M$, its \emph{Hilbert basis} is the set of \emph{indecomposable elements}, i.e., $\Hilb(M)=M^*\setminus(M^*+M^*)$
\end{definition}

\section{Decomposition length}\label{Decomposition_length}

Let $M$ be a numerical monoid generated by a set $\{m_1, \ldots, m_e\}$ and $m_1 < \cdots < m_e$. For any element $m \in M$ and a representation
$m = a_1m_1 + \cdots + a_em_e$, the $e$-tuple $(a_1, \ldots, a_e)$ is a \textit{decomposition of $m \in M$}.

Next we introduce the \emph{maximal decomposition lengths}. For any $m \in M \setminus \{0\}$, a representation $m = \sum_{i = 1}^{e}a_im_i$, with $a \in \Z_+$, is called a \textit{maximal decomposition} if for any other representation $m = \sum_{i = 1}^{e} b_im_i$, with $b_i \in \Z_+$, one has
\[
  \sum_{i = 1}^{e}b_i \leq \sum_{i = 1}^{e} a_i.
\]
Correspondingly, the \emph{maximal decomposition length}, or simply the \textit{length}, of an element $m \in M \setminus \{0\}$ is defined by
\[
  \textbf{l}(m) = \sum_{i = 1}^{e} a_i,
\]
where $m = \sum_{i = 1}^{e}a_im_i$ is any maximal decomposition of $m \in M$. A \textit{maximal decomposition of $m \in M$} is the $e$-tuple $(a_1, \ldots, a_e)$, whereas $\sum_{i = 1}^{e} a_im_i$ is a maximal decomposition. 

For the number of elements in $M$ that have a particular length $k \in \N$ we write
\[
  d_k(M) = \#\{m \in M \mid \textbf{l}(m) = k\}.
\]

For the set of all maximal decompositions for a length $k \in \N$, we write
\[
  \dec_k (M) = \{(a_1, \ldots, a_e) \mid m  = \sum_{i = 1}^{e} a_im_i\ \text{and}\ \textbf{l}(m) = k\}.
\]

Finally, we put

\[
  \textbf{dec}(M) = \bigcup_{k = 1}^{\infty} \textbf{dec}_{k}(M).
\]

Next proposition explain that the numbers $d_k(M)$ and $\dec_k(M)$ are independent of the choices of the generating set $\{m_1,\ldots,m_e\}$.

\begin{proposition}\label{Invariant}
For a numerical monoid $M$ and a natural number $k$, the numbers $d_k(M)$ and $\#\dec_k(M)$ with respect to a generating set $\{m_1,\ldots,m_e\}$ are the same as with respect to $\Hilb(M)$.
\end{proposition}

\begin{proof}
For every element $m\in M$, in any representation of $m$ in terms of the generators $m_1,\ldots,m_e$ we can further decompose the summands into elements of $\Hilb(M)$. On the other hand, by Lemma \ref{system of generators}, $\Hilb(M)\subset\{m_1,\ldots,m_e\}$. This shows that the number $d_k(m)$ is computed in terms of representations of $m$ via $\Hilb(M)$ and, also, in any maximal length representation in terms of $m_1,\ldots,m_e$ only indecomposable elements are used. 
\end{proof}

The numerical sequences $\{d_k(M)\}_{k=0}^\infty$, and $\{\#\dec_k(M)\}_{k=0}^\infty$ are our primary objects of study.

The following lemma is a special case -- the rank one case -- of the \emph{Gordan Lemma}, concerning \emph{affine submonoids} of arbitrary rank \cite[Chapter 2]{Kripo}.

\begin{lemma}\label{Gordan}
Let $M\subset\ZZ_+$ be a numerical monoid and $m=\min(M\setminus\{0\})$. Then $\#\Hilb(M)<m$. In particular, $M$ is finitely generated.
\end{lemma}

\begin{proof}
Since $\#(\ZZ_+\setminus M)<\infty$, every residue class $\mod m$ occurs in $M$. Let $m=m_0,m_1,\ldots,m_{m-1}$ be the smallest elements of $M$, satisfying $m_i=i\mod m$. Then every element $n\in M$ can be (uniquely) written as
\begin{align*}
n=q m+m_i,\quad q\in\ZZ_+,\quad n=i\mod m.
\end{align*} 
In particular, $M=\langle m_0,m_1,\ldots,m_{m-1}\rangle$.
\end{proof}

\section{Frobenius number}\label{FBN}

The Frobenius number of a numerical monoid $M$ is an active research area \cite{Frobenius_book}. The main problem is to express $\F(M)$, or at least an optimal upper bound in terms of a given generating set of $M$. Below we give a short synopses of some of the highlights in the field. For $m_1,\ldots,m_e\in\NN$ with $\gcd(m_1,\ldots,m_e)=1$ and $m_1 < \cdots < m_e$, we denote
$$
\F(m_1,\ldots,m_e)=\F(\langle m_1,\ldots,m_e \rangle).
$$
\begin{enumerate}[label=(\alph*)]
\item $\F(m_1,m_2)=m_1m_2-m_1-m_2$ and there are $\frac{1}{2}(m_1 - 1)(m_2 - 1)$ non-representable positive integers;
\item $\F(m_1,\ldots,m_e)<m_1m_e$;
\item The lower bound for the Frobenius number with embedding dimension 3 is given by \[F(m_1, m_2, m_3) + m_1 + m_2 + m_3 \geq \sqrt{m_1m_2m_3};\]
\item For the following arithmetic sequence we have \[F(m_1, m_1 + m_2, m_1 + 2m_2, \ldots, m_1 + m_2m_3) = \left(\left \lfloor \frac{m_1 - 2}{m_3} \right \rfloor \right)m_1 + m_2(m_1 - 1);\]
\item For the following geometric sequence we have
\begin{align*}
F(m_1^k, m_1^{k-1}m_2, m_1^{k-2}m_2^2, \ldots, m_2^k) &= \\
m_2^{k-1} (m_1m_2 - m_1 - m_2) &+ \frac{m_1^2(n-1)(m_1^{k-1} - m_2^{m_2-1})}{m_1 - m_2};
\end{align*}
\item \textit{Arnold's conjecture.} Let $T = \sum_{i = 1}^e |m_i|$. Thus $T$ is the $\ell^1$ norm of the generators. One of the most famous open problems in the field, Arnold's conjecture, says that $\F(M)$ increases like $T^{1 + 1/(e - 1)}$ \cite[p. 526]{expected_number}. Arnold also conjectured that for ``average behavior'', $\F(M)$ is
\[
    F(M) \sim (e - 1)!^{\frac{1}{e - 1}}(m_1m_2 \cdots m_e)^{\frac{1}{e - 1}} \quad \text{(see \cite[p. 526]{expected_number})}.
\]
\end{enumerate}