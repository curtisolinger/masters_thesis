\chapter{Generating the $d_k(M)$}

We keep the notation, introduced in Section \ref{Decomposition_length}.

Given a natural number $n$, in this section we describe an algorithm for computing the numbers $d_1(M),d_2(M),\ldots,d_n(M)$.

\section{Algorithm 1.}\label{Algorithm 1}

\medskip\noindent\emph{Step 1.} Declare a struct called node to encapsulate a monoid element's value and length, which is not necessarily its maximal decomposition length. This node will be used in a linked list which is generated in the next step. 

\medskip\noindent\emph{Step 2.} To compute the monoid with generating set $\{m_1, \ldots, m_e\}$, create $e$ for-loops where each loop ranges from 0 to $\frac{nm_e}{m_1}$. The nested loops will create duplicate elements with different coefficients and, not necessarily, different lengths (an element can have more than one representation with the same length). 

It is important to create enough copies of each element less than $nm_e - m_1$ so that element's maximum decomposition is found. An element less than $nm_e - m_1$ can be generated with coefficients whose sum is less than or equal to $n$, but its maximum decomposition is not found. 

During the loop, check if the element is less than $nm_e - m_1$ before creating a node and inserting into the linked list. 

\medskip\noindent\emph{Step 3.} Create an array of length $nm_e - m_1$. Hash the linked list into an array where each element of the array corresponds to an element in the linked list. Before inserting a node into the array, check if !NULL and if the new node has a length greater than the existing node. If NULL, insert regardless. Thus the elements of the array will either be NULL or contain a pointer to one node with the maximum decomposition length of that element. 

\begin{verbatim}
LET m = monoid element
IF (m'th element of arr is NULL)
    ASSIGN node to array
    Set next to NULL
ELSE IF (length of node > length of node in array m'th element)
    FREE node in m'th position
    ASSIGN node to array
    Set next to NULL
\end{verbatim}


\medskip\noindent\emph{Step 4.} Create decomposition array of size $n$. Hash previous array into decomposition array by creating a linked list of nodes who all have a maximal decomposition length equal to the decomposition array's element position. 

\begin{verbatim}
IF (element array is not NULL and max decomp length is less than n)
    ASSIGN node in decomposition array to new node next
    ASSIGN new node to decomposition array in length position in array
\end{verbatim}

The number of nodes in the linked list in each position $k$ of the decomposition array will equal $d_k(M)$. Count and print those lists and you are done. 

\section{Algorithm 2}\label{Algorithm 2}


\medskip\noindent\emph{Step 1.} To reduce the computations, one can first extract $\Hilb(M)$ from the generating set $\{m_1,\ldots,m_e\}$. This can be done using \textsf{Normaliz}. The steps below are independent of this steps, though.

\medskip\noindent\emph{Step 2.} Define $M_k=[1,k]\cap M$ and generate the ascending chain of sets
\begin{align*}
M_1\subset M_2\subset M_3\subset\ldots\subset M_{nm_e-m_1},
\end{align*}
inductively as follows. Without loss of generality we can assume $m_1>1$, for otherwise $M=\ZZ_+$ and everything trivializes. Put $M_1=\{0\}$ and, for $k>1$,
\begin{align*}
M_k=
\begin{cases}
M_{k-1},\ \ \text{if}\ \ \big(k-\Hilb(M)\big)\cap M_{k-1}=\emptyset,\\
M_{k-1}\cup\{k\},\ \ \text{if}\ \ \big(k-\Hilb(M)\big)\cap M_{k-1}\not=\emptyset.
\end{cases}
\end{align*}

\medskip\noindent\emph{Step 3.} We construct the subsets $\dec_k(M)\subset\ZZ^e_+$ inductively. Put
\begin{align*}
\dec_1(M)=\{\e_1,\ldots,\e_e)\ \text{and}\ d_1=e.
\end{align*}

Assume we have generated $\dec_k(M)$ for some $k\ge1$. Then one generates $\dec_{k+1}(M)$ as follows.

By Lemma \ref{subsum}, we have the inclusion
\begin{align*}
\dec_{k+1}(M)\subset\bigcup_{i=1}^e\big(\e_i+\dec_k(M)\big)
\end{align*}
For every element $x\in\bigcup_{i=1}^e\big(\e_i+\dec_k(M)\big)$,
one verifies the inclusion $x\in\dec_{k+1}(M)$ by checking the following condition:
\begin{align*}
\big\{\Gamma(x)-\Gamma(y)\ |\ y\in\dec_k(M)\big\}\cap\big(M_{(k+1)m_e-km_1}\setminus\Hilb(M)\big)=\emptyset,
\end{align*} 
the map $\Gamma:\ZZ_+^e\to\ZZ_+$ as in Section \ref{Stabilization}.

\medskip\noindent\emph{Step 4.} After generating the sets $\dec_1(M),\ldots,\dec_n(M)$, the numbers $d_k(M)$ are determined by
\begin{align*}
d_k(M)=\#\left\{\Gamma(x)\ |\ x\in\dec_k(M)\right\}.
\end{align*}

