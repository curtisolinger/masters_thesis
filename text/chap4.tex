\chapter{The main theorem}

\label{main_result}

Throughout this section we assume $M\subset\ZZ_+$ is a numerical monoid, generated by coprime numbers $m_1,\ldots,m_e\in\NN$, satisfying $m_1 < \cdots < m_e$. We follow the notation introduced in Sections \ref{Decomposition_length}.

\section{Associated graded ring of a numerical monoid ring}

Let $\kk$ be a field. Denote by $\gr(\kk[M])$ the associated graded ring with respect to the maximal monomial ideal $I\subset\kk[M]$, i.e., the ideal, generated by $M\setminus\{1\}$. (Recall, in monoid rings we use multiplicative notation for the monoid operation.)

\newpage

\begin{proposition}\label{associated}
\leavevmode
\begin{enumerate}[label=(\alph*)]
\item
We have the graded structure
$$
\gr(\kk[M])\cong\kk\oplus A_1\oplus A_2\oplus\cdots,
$$
making $\gr(\kk[M])$ into a homogeneous graded algebra, i.e., $\gr(\kk[M])$ is generated in degree one, i.e., $\gr(\kk[M])=\kk[A_1]$, where the homogeneous components are:
\begin{align*}
A_k=\bigoplus_{
{
\begin{matrix}
m\in M\setminus\{1\}\\
\l(m)=k
\end{matrix}
}
}
\kk m
\end{align*}
\item For every $m,n\in M\setminus\{1\}$, their product $m*n$ in $\gr(\kk[M])$ is given by
\begin{align*}
m*n=
\begin{cases}
mn,\ \text{if}\ \textbf{l}(m)+\textbf{l}(n)=\textbf{l}(mn)\\
0, \text{if}\ \textbf{l}(m)+\textbf{l}(n)<\textbf{l}(mn).
\end{cases}
\end{align*}
\item The ring $\kk[\ZZ_+m_1]$ ($\cong\kk[X]$) is a graded $\kk$-retract of $\gr(\kk[M])$, i.e., $\kk[\ZZ_+m_1]$ is a sub-algebra of $\kk[M]$ and there is a $\kk$-algebra homomorphism $\kk[M]\to\kk[\ZZ_+m_1]$, which restricts  to the identity map on $\kk[\ZZ_+m_1]$; moreover, $\ker(\gr(\kk[M])\to\kk[\ZZ_+m_1])$ is the nilradical of $\gr(\kk[M])$.
\end{enumerate}
\end{proposition}
 
\begin{proof}
The parts (a, b) follow from the definition of $\gr(\kk[M])$ and the equality $I^k\cap M=\{x_1\cdots x_k|\ x_1,\ldots,x_k\in M\}$ for every $k\in\NN$.

\medskip\noindent(c) That $\kk[\ZZ_+m_1]$ is a $\kk$-vector subspace of $\gr(\kk[M])$ is clear. That the multiplicative structures also agree follows from the part (b) and the observation that $\textbf{l}(m_1^k)=k$ for every $k\in\NN$. We define the map $f:\gr(\kk[M])\to\kk[\ZZ_+m_1]$ to be the $\kk$-algebra homomorphism, defined by
\begin{align*}
f(m)=
\begin{cases}
m\ \ \text{if}\ \ m\in\ZZ_+m_1,\\
0\ \ \text{if}\ \ m\in M\setminus\ZZ_+m_1.
\end{cases}
\end{align*}

To see that $f$ is well defined and $\ker(f)=\sqrt{0}\subset\gr(\kk[M])$ we observe that every element $m\in M\setminus\ZZ_+m_1$ is nilpotent in $\gr(\kk[M])$. In fact, the degree of $m$, viewed as an element of $\gr(\kk[M])$, is $\textbf{l}(m)<\big\lfloor\frac m{m_1}\big\rfloor$. We claim $m^{*m_1}=0\in\gr(\kk[M])$, where the star stands for the exponentiation in the algebra $\gr(\kk[M])$. In fact, if $m^{*m_1}\in\gr(\kk[M])\setminus\{0\}$ then we can write
\begin{align*}
m>\deg(m^{*m_1})=\deg(m_1^{*m})=m,
\end{align*}
a contradiction. 
\end{proof}
\noindent(For a similar description of $\gr(\kk[N])$ for the \emph{affine monoids} of high rank, see \cite{Bottom}.)

\begin{corollary}\label{associated_corollary}
\leavevmode
\begin{enumerate}[label=(\alph*)]
\item As a $\kk$-vector space, $\gr(\kk[M])$ can be thought of as the same $\kk[M]$, where the product in $\gr(\kk[M])$ is the modification of that in $\kk[M]$ according to Proposition \ref{associated}(b).
\item 
The Hilbert function $H:=H_{\gr{(\kk[M]})}$ is given by $H(k)=d_k(M)$, $k\in\NN$.
\item The sequence $d_k(M)$ is eventually constant.
\end{enumerate}
\end{corollary}

\begin{proof}
The part (b) is immediate from Proposition \ref{associated}(a). The part (c) follows from  Lemma \ref{dimension1} and Proposition \ref{associated}(c). 
\end{proof}

\begin{example} Neither the sequence $d_k(M)$ nor even the ring $\gr(\kk[M])$ itself determines the monoid $M$. In fact, for the numerical monoids $M_1=\ZZ_+2+\ZZ_+3$ and $M_2=\ZZ_+2+\ZZ_+5$, the algebras $\gr(\kk[M_1])$ and $\gr(\kk[M_2])$ are both graded isomorphic to $\kk[X,\epsilon]/(\epsilon^2=0)$, where $\deg X=\deg\epsilon=1$.
\end{example}

\section{The second Frobenius number}\label{Stabilization}

In view of Corollary \ref{associated_corollary}(c) we can introduce the following

\begin{definition}\label{2nd_frob}
The \emph{second Frobenius number} of $M$, denoted by $\F\rq{}(M)$ is the smallest natural number $k$, such that $d_{k\rq{}}(M)=d_{k\rq{}+1}(M)$ for every $k\rq{} \geq k$. 
\end{definition}

Corollary \ref{associated_corollary}(b) explains why this definition is analogous to the classical Frobenius number $\F(M)$: the latter can be defined as the smallest natural number $k$, such that $H_{\kk[M]}(k\rq{})=H_{\kk[M]}(k\rq{}+1)$, where $H_{\kk[M]}$ is the Hilbert function of $\kk[M]$ with respect of the standard grading $\deg(m)=m$ for the elements $m\in M$.

Our goal is to give an explicit upper bound for $\F\rq{}(M)$, not accessible via the theory of Hilbert functions of graded algebras.

\newpage

\begin{lemma}\label{just_Hilbert}
For every element $(a_1,\ldots,a_e)\in\dec(M)$, $a_i>0$ implies $m_i\in\Hilb(M)$.
\end{lemma}

\begin{proof}
This is equivalent to the claim that, for a maximal decomposition $m=\sum_{i=1}^e a_im_i$ and an index $1\le i\le e$, the inequality $a_i>0$ implies $m_i\in\Hilb(M)$. In fact, if $m_i\notin\Hilb(M)$ then $m_i$ is a nontrivial positive integer combination of elements of $\Hilb(M)\subset G$, contradicting the assumption that $\sum_{i=1}^e a_im_i$ is a maximal decomposition.
\end{proof}

As a corollary the numbers $d_k(M)$ are independent of the choice of the generating set $G$ of $M$: they only depend on $\Hilb(M)$.

\begin{lemma}\label{subsum}
Assume $(a_1,\ldots,a_e)\in\dec(M)$ and $(b_1,\ldots,b_e)\in\ZZ_+^e\setminus\{0\}$ satisfy $b_i\le a_i$ for $i=1,\ldots,e$. Then $(b_1,\ldots,b_e)\in\dec(M)$.
\end{lemma}

\begin{proof}
This is equivalent to the claim that, for a maximal length decomposition $\sum_{i=1}^e a_im_i$, any non-zero sub-sum $\sum_{i=1}^e b_im_i$ (i.e., $b_i\le a_i$ for every index $i$) is also a maximal length decomposition. In fact, if $\sum_{i=1}^e b_im_i=\sum_{i=1}^e b\rq{}_im_i$ for some $b\rq{}_i\in\ZZ_+$ with $\sum_{i=1}^e b_i<\sum_{i=1}^e b\rq{}_i$, then $\sum_{i=1}^e a_im_i=\sum_{i=1}^e(a_i-b_i+b\rq{}_i)m_i$. Since $a_i-b_i+b\rq{}_i\in\ZZ_+$ and $\sum_{i=1}^e a_i<\sum_{i=1}^e(a_i-b_i+b\rq{}_i)$, this contradicts the assumption that $\sum_{i=1}^e a_im_i$ is a maximal length decomposition.
\end{proof}

%\newpage

\begin{theorem} \label{stabilization}
\leavevmode
\begin{enumerate}[label=(\alph*)]
\item  For every natural number $k\ge(e-1)m_1$, one has
\begin{align*}
\#\dec_k(M)\ge\#\dec_{k+1}(M)\quad\text{and}\quad d_k(M)\ge d_{k+1}(M).
\end{align*}
\item For every natural number $k\ge(e-1)m_e$, one has
\begin{align*}
\#\dec_k(M)=\#\dec_{k+1}(M)\quad\text{and}\quad d_k(M)=d_{k+1}(M).
\end{align*}
In particular $\F\rq{}(M) \leq (e-1)m_e$.
\item For every pair of natural numbers $a,b>(e-1)m_1m_e$, one has $a, b \in M$ and
\begin{align*}
\textbf{l}(a)=\textbf{l}(b)\quad\Longleftrightarrow\quad\bigg\lfloor{\frac a{m_1}}\bigg\rfloor=\bigg\lfloor{\frac b{m_1}}\bigg\rfloor.
\end{align*}
In particular $d_k(M)=m_1$ for $k\ge(e-1)m_e$.
\end{enumerate}

\end{theorem}

\begin{remark}
(a) Although it follows from Theorem \ref{stabilization}(c) that $\F(M)\le(e-1)m_1m_e$, a much better bound for $\F(M)$ is known, namely $\F(M)<m_1m_e$ (Section \ref{FBN}). On the other hand, the periodic behavior in Theorem \ref{stabilization}(c) is more than an upper bound for $\F(M)$. 

\medskip\noindent(b) It follows from \cite{maximal_length} that the function $\textbf{l}(-)$ is eventually of the form in Theorem \ref{stabilization}(c), but we also  give an explicit lower bound from where this behavior shows up.
\end{remark}

We will need two lemmas.

\begin{lemma}\label{bounded}
For every $(a_1,\ldots,a_e)\in\dec(M)$ and $i\in\{2,\ldots,e\}$, one has $a_i<m_1$.
\end{lemma}

\begin{proof}
Assume to the contrary $a_j\ge m_1$ for some $j\ge2$. Then we can write
\begin{align*}
&a_1m_1+\cdots+a_em_e=\\
&a\rq{}_1m_1+a_2m_2+\cdots+a_{j-1}m_{j-1}+a\rq{}_jm_j+a_{j+1}m_{j+1}+\cdots+a_em_e,
\end{align*}
where $a\rq{}_1=a_1+m_j$ and $a\rq{}_j=a_j-m_1$. This contradicts the containment $(a_1,\ldots,a_e)\in\dec(M)$ because $a_1+a_j<a'_1+a'_j$.
\end{proof}

To state the second lemma, we first introduce the following objects:
\begin{enumerate}[label=(\alph*)]
\item For every $k\in\ZZ_+$, the affine hyperplane and affine half-space:
\begin{align*}
\G_k:=(X_1+\cdots+X_e=k)\ \subset\ \RR^e,\\
\G_k^-:=(X_1+\cdots+X_e\le k)\ \subset\ \RR^e;
\end{align*}
\item The two infinite right prisms:
\begin{align*}
\Pi_+=\{(x_1,x_2,\ldots,x_e)\ |\ &x_1,x_2,\ldots,x_e\ge0\ \ \text{and}\\ 
&x_2+\cdots+x_e\le (e-1)m_1\}\subset\RR_+^e,
\end{align*}
and
\begin{align*}
\Pi=\{(x_1,x_2,\ldots,x_e)\ |\ &x_2,\ldots,x_e\ge0\ \ \text{and}\\ 
&x_2+\cdots+x_e\le (e-1)m_1\}\subset\RR\times\RR_+^{e-1};
\end{align*}
\item
The linear map 
\begin{align*}
\Gamma:\RR^e&\to\RR,\\
(x_1,\ldots,x_e)&\mapsto\sum_{i=1}^em_ix_i;
\end{align*} 
\item
The sequence of affine hyperplanes $\{\H_s\}_{i=0}^\infty$ in $\RR^e$, defined by the following conditions:
\begin{enumerate}[label=(\alph*)]
\item
each $\H_s$ is parallel to the hyperplane $\ker\Gamma$,
\item For every $s$, the set $\H\cap\ZZ^e_+$ is not empty,
\item
the distances $\delta_s$ between $\H_s$ and $0$ form a strictly increasing sequence of non-negative real numbers,
\item 
$\ZZ_+^e\subset\bigcup_{i=0}^\infty\H_s$;
\end{enumerate}
\item
The sequence of lattice polytopes
\begin{align*}
P_s=\conv(\H_s\cap\Pi_+\cap\ZZ^e)\subset\RR_+^e,\qquad s=0,1,\ldots
\end{align*}
(for some initial vlues of $s$ the polytope $P_s$ may be empty).
\end{enumerate}

For $s\in\ZZ_+$, the \emph{$s$-th element of $M$} refers to the $s$-the element of $M$ in the natural order.

\begin{lemma}\label{stratification}
\leavevmode
\begin{enumerate}[label=(\alph*)]
\item $\{\delta_s\}_{s=0}^\infty$ is additive submonoid of $\RR_+$, the non-negative reals, and isomorphic to $M$;
\item The set of presentations $m=\sum_{i=1}^ea_im_i$ with $a_i\in\ZZ_+$ of the $s$-th element $m\in M$ is bijective to $\H_s\cap\ZZ^e_+$;
\item For every $k\in\NN$, $d_k(M)$ equals the number of those indices $s$, for which $P_s\cap\G_k\not=\emptyset$ and $P_s\subset\G_k^-$. 
\end{enumerate}
\end{lemma}

\begin{proof} Part (a) follows from the definition of the $\H_s$, Part (b) is a consequence of (a), and Part (c) is a consequence of (b) in view of Lemma \ref{bounded} and the inclusion
\begin{align*}
\{(a_1,\ldots,a_e)\ |\ a_1,\ldots,a_e\in\ZZ_+\ \text{and}\ a_2,\ldots,a_e\le m_1-1\}\ \subset\ \Pi_+.
\end{align*}
\end{proof}

\medskip\noindent\emph{Notice.} Instead of $\Pi_+$ and $\Pi$ we could have chosen the narrower prisms, defined by the inequalities $x_2+\cdots+x_e\le(e-1)(m_1-1)$. This would result in minor improvements of the bounds in Theorem \ref{stabilization}(b, c), but at the expense of complicated quotient expressions for the lower bounds instead of natural numbers.

\begin{proof}[Proof of Theorem \ref{stabilization}]
(a) For a natural number $k>(e-1)m_1$ and an element $(a_1,\ldots,a_e)\in\dec_k(M)$, Lemma \ref{bounded} implies $a_1>0$. Consequently, for every $k\ge(e-1)m_1$, Lemma \ref{subsum} implies the injective map
\begin{align*}
\iota_{k+1}:\dec_{k+1}(M)&\to\dec_k(M),\\
(a_1,a_2,\ldots,a_e)&\mapsto(a_1-1,a_2,\ldots,a_e).
\end{align*}
This proves the inequalities for $\#\dec_k(M)$.

For every $k\in\NN$, we have $d_k(M)=\#\Gamma\big(\dec_k(M)\big)$. Consequently, the inequalities for $d_k(M)$ follow from the observation that, for an index $k\ge(e-1)m_1$ and elements $(a_1,\ldots,a_e),(b_1,\ldots,b_e)\in\dec_{k+1}(M)$, the following implication holds: 
\begin{align*}
\Gamma\big((a_1,\ldots,a_e)\big)&\not=\Gamma\big((b_1,\ldots,b_e)\big)\ \Longrightarrow\\
\Gamma\big(\iota_{k+1}(a_1,\ldots,a_e)\big)&\not=\Gamma\big(\iota_{k+1}(b_1,\ldots,b_e)\big).
\end{align*} 

\medskip(b) Fix a natural number $k\ge(e-1)m_e$. Since $m_1$ is the smallest generator of $M$ we have $k\e_1\in\dec_k(M)$. It represents the element $km_1\in M$. Assume $ke_1=\Gamma(\H_s\cap\ZZ_+^e)$ for some $s\in\NN$, i.e., $km_1$ is the $s$-th element in $M$. 

We claim that
\begin{equation}\label{positive}
\H_s\cap\Pi_+=\H_s\cap\Pi.
\end{equation}

This equality is equivalent to the condition that, for every $i\in\{2,\ldots,e\}$, the first coordinate of the point
\begin{align*}
\H_s\cap\big((e-1)m_1\e_i+\RR\e_1\big)
\end{align*}
is non-negative, or equivalently, for every $i\in\{2,\ldots,e\}$, the first coordinate of the point
\begin{align*}
\H_0\cap\big((e-1)m_1\e_i+\RR\e_1\big)=\big(-k\e_1+\H_s\big)\cap\big((e-1)m_1\e_i+\RR\e_1\big)
\end{align*}
is at least $-(e-1)m_e$. The mentioned point is the solution to the equation
$$
m_1x_1+\cdots+m_ex_e=0,
$$
subject to $x_j=0$ for $j\not=1,i$ and $x_i=(e-1)m_1$. The first coordinate of this points is $-(e-1)m_i\ge-(e-1)m_e$.
This proves (\ref{positive}).

For every element $m\in M$ with $\textbf{l}(m)=k$ we have $km_1\le m$, implying $m=\Gamma(\H_t\cap\ZZ_+^e)$ for some $t>s$. In particular, Lemma  \ref{stratification}(c) implies that, for every index $t$ with $P_t\cap\G_k\not=0$ and $P_t\subset\G_k^-$, one has $t>s$ and, therefore, (\ref{positive}) implies $\H_t\cap\Pi=\H_t\cap\Pi_+$. This, in turn, implies the following.

\medskip\noindent{\textbf{Claim}.} The set of hyperplanes $\H_t$, satisfying $P_t\cap\G_{k+1}\not=0$ and $P_t\subset\G_{k+1}^-$ is the parallel translate by $\e_1$ of the set of hyperplanes $\H_{t\rq{}}$, satisfying $P_{t\rq{}}\cap\G_k\not=0$ and $P_{t\rq{}}\subset\G_k^-$. 

\medskip Lemma \ref{stratification} and the Claim together imply Part (b). 

\medskip(c) Lemma \ref{stratification} and the Claim above also imply that, starting from the $((e-1)m_e)$-th member, the monoid $M$ exhibits a periodic pattern: denoting by $\mu$ this member, a natural number $n\ge\mu$ is in $M$ if and only if $n+h\in M$, where
\begin{equation}\label{h}
h=\frac{\Gamma(\e_1)}{\min(\Gamma(\ZZ^e)\cap\NN)}=\frac{m_1}{\gcd(m_1,\ldots,m_e)}=m_1.
\end{equation}
Since $n\in M$ for $n\gg0$, one has $\F(M)<\mu$. But, obviously, $\mu\le(e-1)m_em_1$.

As above, we assume $k\e_1\in\H_s$. For every $t\ge s$, the equality (\ref{h}) implies 
\begin{equation}\label{less}
\H_t\cap\RR\e_1=\frac{t-s+1}{m_1}\cdot\e_1.
\end{equation}
For the equalities in Part (c), one needs to show that $P_t\cap\G_k\not=0$ and $P_t\subset\G_k^-$ for every index $s<t\le s+m_1-1$. Assume to the contrary this is not the case for some index $t$. Then, $P_t\cap\G_r\not=0$ and $P_t\subset\G_r^-$ for some $r>k$. By Sublemma, $\H_t$ is the parallel translate by $(r-k)\e_1$ of some $\H_{t\rq{}}$ with $s\le t\rq{}<t$. We arrive at the contradiction
\begin{align*}
\frac{t-t\rq{}}{m_1}=\frac{t-s+1}{m_1}-\frac{t\rq{}-s+1}{m_1}=r-k \ge 1,
\end{align*}
where the second equality is due to (\ref{less}).
\end{proof}