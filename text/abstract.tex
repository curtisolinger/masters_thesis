% (This file is included by thesis.tex; you do not latex it by itself.)
\begin{abstract}

% The text of the abstract goes here.  If you need to use a \section
% command you will need to use \section*, \subsection*, etc. so that
% you don't get any numbering.  You probably won't be using any of
% these commands in the abstract anyway.

Numerical monoids, i.e., the additive submonoids of non-negative integers with finite complement, are central objects of study at the crossroads of combinatorics, additive number theory, and commutative algebra. A well-known invariant, attached to a numerical monoid, is the Frobenius number. It is the largest integer not in the monoid. Despite many partial results, no general formula is known for the Frobenius number in terms of given generators of the monoid. We introduce a new invariant of a numerical monoid, called the second Frobenius number. It carries important structural information on the monoid. A commutative algebra perspective explains the relationship between the original and the second Frobenius numbers: both are the threshold points for stabilization of Hilbert functions – for the monoid ring itself in the classical setting  and for the associated graded algebra in our setting. Our main theoretical result is an explicit upper bound for the second Frobenius number in terms of given generators of the monoid. We also develop several algorithms for computing the second Frobenius number and present many computational results, strongly suggesting that the Hilbert functions may be stabilizing much faster than our theoretical bound.

\end{abstract}
