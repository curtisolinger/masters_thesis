\chapter{Algorithms}

\section{Algorithm 1}

\begin{lstlisting}[language=C]  
#include <stdio.h>
#include <stdlib.h>
#include <stdbool.h>
#include <time.h>
#include <string.h>

// Define the max element length
const int N = 101;

struct node {
    int element;
    int coef[5];
    int length;
    struct node *next;
};

void hash_monoid(struct node **arr, struct node *list, int M_MAX, int E);
void hash_element_arr(struct node **arr0, struct node **arr1, int M_MAX);
void free_list(struct node *list);
int size(struct node *ptr);
void bubble_sort(int *arr, int n);
int gcd(int a, int b);
int findGCD(int *arr, int n);
void print_gen_set(int *arr, int E);
void create_gen_set(int *arr, int max, int min, int E);
void free_dec(struct node **n);
struct node *create_monoid(int *arr, int M_MAX, int M, int E);
void free_hash(struct node **arr, int M_MAX);
void print_to_file(struct node **arr0, int *arr1, int E);

int main(void)
{
    // Declare the embedding dimension variable E
    int E;

    // Define the upper and lower bounds for the generators.
    int max = 300;
    int min = 50;

    // Prompt the user for the number of generators
    printf("Program to calculate d_k(M) for k = {0, ..., %i}. The generators will be between %i and %i inclusive.\n", N - 1, min, max);
    printf("Please enter in the number of generators: ");
    scanf("%i", &E);

    clock_t start, end;
    double cpu_time_used;
    start = clock();

    // Define an array to hold the E generators
    int gen_set[E];

    // Create the set of generators, check if they are coprime, and if they are not, divide by the gcd.
    create_gen_set(gen_set, max, min, E);

    // Print to screen the generators
    print_gen_set(gen_set, E);

    // Define the max element in the numerical monoid
    const int M_MAX = (N*gen_set[E - 1]) - gen_set[0] + 1;
    printf("M_MAX: %i\n", M_MAX);

    // Define the number of times to loop through each generator.
    // LOOP_MAX must be large enough so that numbers less than M_MAX have their max decomposition calculated. 
    // E.G. If M = <2,3>, then M_MAX = 302. Notice 298 = 149*2 + 0*3. Thus 298 has length of 149. But that wont be 
    // calculated if loop only goes through N. So LOOP_MAX = 151. 
    const int LOOP_MAX = N*gen_set[E-1] / gen_set[0];
    printf("LOOP_MAX: %i\n", LOOP_MAX);

    // Build the monoid from 0 to M_MAX
    struct node *monoid = create_monoid(gen_set, M_MAX, LOOP_MAX, E);
    
    // Hash monoid by element length, only keeping max decomposition length.
    // The i'th element of the array contains a pointer to the node with the maximum decomposition of element i. 
    struct node *element_arr[M_MAX];
    hash_monoid(element_arr, monoid, M_MAX, E);

    // Hash element_arr by length. The i'th element of the array contains a pointer to a linked list of nodes
    // who all have a length of i. 
    struct node *length_arr[N];
    hash_element_arr(length_arr, element_arr, M_MAX);

    // Print k and d_k(M) to csv file 
    print_to_file(length_arr, gen_set, E);

    free_dec(length_arr);
    end = clock();
    cpu_time_used = ((double) (end - start)) / CLOCKS_PER_SEC;
    printf("Program took %f seconds to execute\n", cpu_time_used);
    return 0;
}

// Function to build the numerical monoid linked list
struct node *create_monoid(int *arr, int M_MAX, int M, int E)
{
    int i, j, k, l, m;
    struct node *n;
    struct node *list;
    list = NULL;
    if (E == 2){
        for (i = 0; i < M; i++){
            for (j = 0; j < M; j++){
                //if (i + j < N) {
                if (i*arr[0] + j*arr[1] < M_MAX){
                    n = malloc(sizeof(*n));
                    if (!n){
                        free_list(list);
                        printf("ERROR\n");
                    }
                    // memset(n, 0, sizeof(*n));
                    n -> element = i*arr[0] + j*arr[1];
                    n -> coef[0] = i;
                    n -> coef[1] = j;
                    n -> length = i + j;
                    n -> next = list;
                    list = n;
                }
            }
        }
    }
    else if (E == 3){
        for (i = 0; i < M; i++){
            for (j = 0; j < M; j++){
                for (k = 0; k < M; k++){
                    if (i*arr[0] + j*arr[1] + k*arr[2] < M_MAX){
                        n = malloc(sizeof(*n));
                        if (!n){
                            free_list(list);
                            printf("ERROR\n");
                        }
                        n -> element = i*arr[0] + j*arr[1] + k*arr[2];
                        n -> coef[0] = i;
                        n -> coef[1] = j;
                        n -> coef[2] = k;
                        n -> length = i + j + k;
                        n -> next = list;
                        list = n;
                    }
                }
            }
        }
    }
    else if (E == 4){
        for (i = 0; i < M; i++){
            for (j = 0; j < M; j++){
                for (k = 0; k < M; k++){
                    for (l = 0; l < M; l++){
                        if (i*arr[0] + j*arr[1] + k*arr[2] + l*arr[3] < M_MAX){
                            n = malloc(sizeof(*n));
                            if (!n){
                                free_list(list);
                                printf("ERROR\n");
                            }
                            n -> element = i*arr[0] + j*arr[1] + k*arr[2] + l*arr[3];
                            n -> coef[0] = i;
                            n -> coef[1] = j;
                            n -> coef[2] = k;
                            n -> coef[3] = l;
                            n -> length = i + j + k + l;
                            n -> next = list;
                            list = n;
                        }
                    }
                }
            }
        }
    }
    else if (E == 5){
        for (i = 0; i < M; i++){
            for (j = 0; j < M; j++){
                for (k = 0; k < M; k++){
                    for (l = 0; l < M; l++){
                        for (m = 0; m < M; m++){
                            if (i*arr[0] + j*arr[1] + k*arr[2] + l*arr[3] + m*arr[4] < M_MAX){
                                n = malloc(sizeof(*n));
                                if (!n){
                                    free_list(list);
                                    printf("ERROR\n");
                                }
                                n -> element = i*arr[0] + j*arr[1] + k*arr[2] + l*arr[3] + m*arr[4];
                                n -> coef[0] = i;
                                n -> coef[1] = j;
                                n -> coef[2] = k;
                                n -> coef[3] = l;
                                n -> coef[4] = m;
                                n -> length = i + j + k + l + m;
                                n -> next = list;
                                list = n;
                            }
                        }
                    }
                }
            }
        }
    }
    else{
        printf("Incorrect number of generators\n");
        printf("ERROR\n");
    }
    return list;
}

// hash_monoid(element_arr, monoid, M_MAX, int E);
void hash_monoid(struct node **arr, struct node *list, int M_MAX, int E)
{
    int i, x;
    struct node *tmp = NULL;
    for (i = 0; i < M_MAX; i++)
        arr[i] = NULL;
    while (list != NULL) {
        tmp = list;
        list = tmp -> next;
        x = tmp -> element;
        if (x < M_MAX && arr[x] == NULL) {
            arr[x] = tmp;
            tmp -> next = NULL;
        }
        else if (x < M_MAX && tmp -> length > arr[x] -> length) {
            free(arr[x]);
            arr[x] = tmp;
            tmp -> next = NULL;
        }
        else {
            free(tmp);
        }
    }
}



// hash_element_arr(length_arr, element_arr, M_MAX);
void hash_element_arr(struct node **arr0, struct node **arr1, int M_MAX)
{
    int i, x;
    struct node *tmp = NULL;
    for (i = 0; i < N; i++)
        arr0[i] = NULL;
    for (i = 0; i < M_MAX; i++){
        if (arr1[i] != NULL){
            tmp = arr1[i];
            x = tmp -> length;
            if (x < N){
                tmp -> next = arr0[x];
                arr0[x] = tmp;
            }
            else{
                free(tmp);
            }
        }
    }
}

int size(struct node *ptr)
{
    int counter = 0;
    struct node *tmp = ptr;
    while (tmp != NULL){
        counter++;
        tmp = tmp -> next;
    }
    return counter;
}

void free_list(struct node *list)
{
    struct node *tmp;
    while (list != NULL){
        tmp = list -> next;
        free(list);
        list = tmp;
    }
}

void free_dec(struct node **n)
{
    struct node *tmp1, *tmp2;
    for (int i = 0; i < N; i++){
        tmp1 = n[i];
        while (tmp1 != NULL){
            tmp2 = tmp1 -> next;
            free (tmp1);
            tmp1 = tmp2;
        }
    }
}

void bubble_sort(int *arr, int n){
    int i = 0, j = 0, tmp;
    for (i = 0; i < n; i++){   // loop n times - 1 per element
        for (j = 0; j < n - i - 1; j++){ // last i elements are sorted already
            if (arr[j] > arr[j + 1]){  // swap if order is broken
                tmp = arr[j];
                arr[j] = arr[j + 1];
                arr[j + 1] = tmp;
            }
        }
    }
}


// Function to return gcd of a and b
int gcd(int a, int b)
{
    if (a == 0)
        return b;
    return gcd(b % a, a);
}

// Function to find gcd of array of
// numbers
int findGCD(int *arr, int n)
{
    int result = arr[0];
    for (int i = 1; i < n; i++){
        result = gcd(arr[i], result);
        if(result == 1){
            return 1;
        }
    }
    return result;
}

void create_gen_set(int *arr, int max, int min, int E)
{
    int i;
    // Fill gen_set with random integers between min and max
    srand(time(NULL));
    for (i = 0; i < E; i ++){
        arr[i] = (rand() % (max - min)) + min;
    }
    // Sort gen_set from minimum value to maximum for readability
    bubble_sort(arr, E);
    // Check with the elements in gen_set are coprime.
    // If they are not, divide all elements by the gcd to get a coprime list.
    int gcd = findGCD(arr, E);
    if (gcd != 1){
        for (i = 0; i < E; i++){
            arr[i] = arr[i] / gcd;
        }
    }
}

void print_gen_set(int *arr, int E)
{
    printf("The generators are: ");
    for (int i = 0; i < E; i ++){
        printf("%i", arr[i]);
        if (i < E - 1){
            printf(", ");
        }
    }
    printf("\n");
}

void free_hash(struct node **arr, int M_MAX) 
{
    struct node *tmp0, *tmp1;
    for (int i = 0; i < M_MAX; i++){
        tmp0 = arr[i];
        while (tmp0 != NULL){
            tmp1 = tmp0 -> next;
            free(tmp0);
            tmp0 = tmp1;
        }
    }
}

void print_to_file(struct node **arr0, int *arr1, int E)
{
    int i;
    char buffer[30];
    sprintf(buffer, "output/lengths_%i.csv", E);
    FILE *file = fopen(buffer, "w");
    if (file == NULL){
        printf("Could not open file\n");
    }
    fprintf(file, "k, d_k(M),");
    for (i = 0; i < E; i++){
        fprintf(file, "%i", arr1[i]);
        if (i < E - 1) {
            fprintf(file, ",");
        }
    }
    fprintf(file, "\n");
    for (i = 0; i < N; i++)
        fprintf(file, "%i, %i\n", i, size(arr0[i]));
    fclose(file);
}
\end{lstlisting}    

\newpage

\section{Algorithm 2}


\begin{lstlisting}[language=C]
#include <stdio.h>
#include <stdlib.h>
#include <stdbool.h>
#include <time.h>
#include <assert.h>
#include <string.h>

// Define the max element length
const int N = 101;

struct node{
    // Define the numerical monoid element
    int element;
    struct node *next;
};

// Define a node for each element in dec_k (M) - the set of coefficients for each maximal decomposition of length k.
struct dec_node{
    // Define an array to hold the coefficients for each monoid element
    int coef[6];
    struct dec_node *next;
};

struct dec_node *create_dec_nodes(struct dec_node *n0, struct dec_node *n1, int *arr0, struct node **arr1, int E, int k);
struct node *create_monoid(struct node *n0, int x, int *arr0, int E);
bool compare(int *x, struct node *n, int E);
void print_MONOID(struct node **n, int M_MAX);
void print_dec_node_list(struct dec_node *n, int E);
void free_MONOID(struct node **n, int M_MAX);
void free_dec(struct dec_node **n);
void print_dec_node(const char *label, struct dec_node *n, int E);
void duplicates(struct dec_node *ptr, int E);
struct dec_node *remove_non_max(struct dec_node *n0, struct dec_node *n1, int *arr0, struct node **arr1, int E, int k);
bool find(int z, struct node *ptr, int *arr, int E);
int gamma(struct dec_node *n, int *arr, int E);
int size(struct dec_node *ptr);
int size_hash(struct node *ptr);
struct node *hash_dec_nodes(struct dec_node *n, int *arr0, int E);
void duplicates_hash(struct node *ptr, int E);
void free_hash(struct node **n);
void bubble_sort(int *arr, int n);
int gcd(int a, int b);
int findGCD(int *arr, int n);
void print_gen_set(int *arr, int E);
void create_gen_set(int *arr, int max, int min, int E);

int main(void)
{
    // Declare the embedding dimension variable E
    int i, E;

    // Define the upper and lower bounds for the generators.
    int max = 300;
    int min = 50;
    
    // Prompt the user for the number of generators
    printf("Program to calculate d_k(M) for k = {0, ..., %i}. The generators will be between %i and %i inclusive.\n", N - 1, min, max);
    printf("Please enter in the number of generators: ");
    scanf("%i", &E);
    
    // Define an array to hold the n generators
    int gen_set[E];

    // Create the set of generators, check if they are coprime,
    // and if they are not, divide by the gcd.
    create_gen_set(gen_set, max, min, E);
    
    // Print out what the random generators are
    print_gen_set(gen_set, E);
    
    // Start programming timming
    clock_t start, end;
    double cpu_time_used;
    start = clock();

    // Define the max element in the numerical monoid
    const int M_MAX = (N*gen_set[E - 1]) - gen_set[0];

    // Declare an array of pointers to monoid elements in struct node and fill with NULL
    struct node *MONOID[M_MAX];
    for (i = 0; i < M_MAX; i++) 
        MONOID[i] = NULL;

    // Inductively create the monoid
    for (int i = 0; i < M_MAX; i++){
        if (i == 0){
            MONOID[0] = malloc(sizeof(struct node));
            if (!MONOID[0]) {
                printf("ERROR: line number %d in function %s\n", __LINE__, __func__);
            }
            memset(MONOID[0], 0, sizeof(*MONOID[0]));
        }
        else{
            MONOID[i] = create_monoid(MONOID[i - 1], i, gen_set, E);
        }
    }

    // Declare an array of pointers to a dec_node linked list and fill with NULL
    struct dec_node *dec[N];
    for (i = 0; i < N; i++) 
        dec[i] = NULL;

    // Build dec_k(M)
    for (i = 0; i < N; i++){
        if (i == 0)
            dec[0] = create_dec_nodes(NULL, NULL, NULL, NULL, E, 0);
        else if (i == 1)
            dec[1] = create_dec_nodes(dec[0], NULL, NULL, NULL, E, 1);
        else
            dec[i] = create_dec_nodes(dec[1], dec[i-1], gen_set, MONOID, E, i);
    }

    // Remove duplicate elements with equal lengths
    struct node *hash_table[N];
    for (i = 0; i < N; i++) 
        hash_table[i] = NULL;
    for (i = 0; i < N; i++)
        hash_table[i] = hash_dec_nodes(dec[i], gen_set, E);

    char buffer[23];
    sprintf(buffer, "output/lengths_%i.csv", E);

    FILE *file = fopen(buffer, "w");
    if (file == NULL){
        printf("Could not open file\n");
        return 1;
    }
    fprintf(file, "k, d_k(M),");
    for (i = 0; i < E; i++) {
        fprintf(file, "%i", gen_set[i]);
        if (i < E - 1) {
            fprintf(file, ",");
        }
    }
    fprintf(file, "\n");
    for (i = 0; i < N; i++)
        fprintf(file, "%i, %i\n", i, size_hash(hash_table[i]));
    fclose(file);

    free_MONOID(MONOID, M_MAX);
    free_dec(dec);
    free_hash(hash_table);

    end = clock();
    cpu_time_used = ((double) (end - start)) / CLOCKS_PER_SEC;

    printf("Program took %f seconds to execute\n", cpu_time_used);
    return 0;
}

void create_gen_set(int *arr, int max, int min, int E)
{
    int i;
    // Fill gen_set with random integers between min and max
    srand(time(NULL));
    for (i = 0; i < E; i ++){
        arr[i] = (rand() % (max - min)) + min;
    }
    
    // Sort gen_set from minimum value to maximum for readability
    bubble_sort(arr, E);
    
    // Check with the elements in gen_set are coprime. 
    // If they are not, divide all elements by the gcd to get a coprime list.
    int gcd = findGCD(arr, E);
    if (gcd != 1){
        for (i = 0; i < E; i++){
            arr[i] = arr[i] / gcd;
        }
    }
}

void print_gen_set(int *arr, int E)
{
    printf("The generators are: ");
    for (int i = 0; i < E; i ++){
        printf("%i", arr[i]);
        if (i < E - 1){
            printf(", ");
        }
    }
    printf("\n");
}


struct dec_node *create_dec_nodes(struct dec_node *n0, struct dec_node *n1, int *arr0, struct node **arr1, int E, int k)
{
    int i, j;
    struct dec_node *n;
    struct dec_node *list = NULL;
    struct dec_node *tmp1;

    if (n0 == NULL){
        list = malloc(sizeof(*list));
        if (!list) {
            printf("ERROR: line number %d in function %s\n", __LINE__, __func__);
        }
        memset(list, 0, sizeof(*list));
    }
    else if (n1 == NULL){
        for (i = 0; i < E; i++){ // Put the right number of zeros at the front of the vector
            n = malloc(sizeof(*n));
            if (!n) {
                printf("ERROR: line number %d in function %s\n", __LINE__, __func__);
            }
            for (j = 0; j < i; j++)
                n -> coef[j] = 0;
            n -> coef[j] = 1; // Put 1 in the right place
            for (j = j + 1; j < E; j++) // Fill in the rest of the zeros
                n -> coef[j] = 0;
            n -> next = list;
            list = n;
        }
    }
    else{
        while (n0 != NULL){
            tmp1 = n1;
            while (tmp1 != NULL){
                n = malloc(sizeof(*n));
                if (!n) {
                    printf("ERROR: line number %d in function %s\n", __LINE__, __func__);
                }
                for (i = 0; i < E; i++)
                    n -> coef[i] = n0 -> coef[i] + tmp1 -> coef[i];
                n -> next = list;
                list = n;
                tmp1 = tmp1 -> next;
            }
            n0 = n0 -> next;
        }
        if (list != NULL) {
            duplicates(list, E);
            list = remove_non_max(list, n1, arr0, arr1, E, k);
        }
    }
    return list;
}

struct dec_node *remove_non_max(struct dec_node *n0, struct dec_node *n1, int *arr0, struct node **arr1, int E, int k)
{
    struct dec_node *tmp0, *tmp1, *prev;
    int M = (k+1)*arr0[E - 1] - k*arr0[0];
    tmp0 = n0;
    tmp1 = n1;

    // If n0 node itself holds the key to be deleted
    while (tmp1 != NULL && tmp0 != NULL){
        if (find(gamma(tmp0, arr0, E) - gamma(tmp1, arr0, E), arr1[M+1], arr0, E)){
            n0 = tmp0 -> next;
            free(tmp0);
            tmp0 = n0;
            tmp1 = n1;
        }
        else{
            tmp1 = tmp1 -> next;
        }
    }

    // Advance tmp0 since the first node is checked
    tmp0 = n0 -> next;
    prev = n0;

    while (tmp0 != NULL){
        tmp1 = n1;
        while (tmp1 != NULL){
            if (find(gamma(tmp0, arr0, E) - gamma(tmp1, arr0, E), arr1[M+1], arr0, E)){
                prev -> next = tmp0 -> next;
                free(tmp0);
                tmp0 = prev -> next;
                tmp1 = n1;
            }
            else{
                tmp1 = tmp1 -> next;
            }
        }
        if (tmp0 != NULL){
            tmp0 = tmp0 -> next;
            prev = prev -> next;
        }
    }
    return n0;
}

struct node *hash_dec_nodes(struct dec_node *n0, int *arr0, int E)
{
    struct dec_node *tmp = n0;
    struct node *list = NULL;
    struct node *n = NULL;
    while (tmp != NULL){
        n = malloc(sizeof(*n));
        if (!n) {
            printf("ERROR: line number %d in function %s\n", __LINE__, __func__);
        }
        n -> element = gamma(tmp, arr0, E);
        n -> next = list;
        list = n;
        tmp = tmp -> next;
    }
    duplicates_hash(list, E);
    return list;
}

void duplicates_hash(struct node *list, int E)
{
    struct node *tmp0, *tmp1, *prev;
    int i, j;
    tmp0 = list;
    while (tmp0 != NULL && tmp0 -> next != NULL){
        prev = tmp0;
        tmp1 = prev -> next;
        while (tmp1 != NULL){
            if (tmp0 -> element != tmp1 -> element) {
                prev = tmp1;
                tmp1 = prev -> next;
            }
            else if (tmp0 -> element == tmp1 -> element) {
                prev -> next = tmp1 -> next;
                free(tmp1);
                tmp1 = prev -> next;
            }
        }
        tmp0 = tmp0 -> next;
    }
}

int gamma(struct dec_node *n, int *arr, int E)
{
    int x = 0;
    for (int i = 0; i < E && n != NULL; i++){
        x += n -> coef[i] * arr[i];
    }
    return x;
}

bool find(int z, struct node *ptr, int *arr, int E)
{
    // Check to see if z is an element in the Hilbert basis. If so, then return false.
    for (int i = 0; i < E; i++){
        if (z == arr[i])
            return false;
    }
    struct node *tmp_1 = ptr;
    while (tmp_1 != NULL){
        if (z == tmp_1 -> element){
            return true;
        }
        tmp_1 = tmp_1 -> next;
    }
    return false;
}

void print_dec_node_list(struct dec_node *n, int E)
{
    // Print out the elements in the i'th dec set
    while (n != NULL){
        printf("(%i", n -> coef[0]);
        for (int i = 1; i < E; i++)
            printf(", %i", n -> coef[i]);
        printf(")  ");
        n = n -> next;
    }
    printf("\n");
}

void print_dec_node(const char *label, struct dec_node *n, int E)
{
    printf("%s: ", label);
    printf("(%i", n -> coef[0]);
    for (int i = 1; i < E; i++)
        printf(", %i", n -> coef[i]);
    printf(")\n");
}

struct node *create_monoid(struct node *n0, int k, int *arr0, int E)
{
    int i;
    int arr1[E];
    struct node *tmp0 = n0;
    struct node *list = NULL;
    struct node *n;

    // Calculate the set k - Hilb(M)
    for (i = 0; i < E; i++){
        // *(arr1 + i) = k - *(arr0 + i);
        arr1[i] = k - arr0[i];
    }

    while (tmp0 != NULL)
    {
        n = malloc(sizeof(*n));
        if (!n){
            printf("ERROR: line number %d in function %s\n", __LINE__, __func__);
        }
        n -> element = tmp0 -> element;
        n -> next = list;
        list = n;
        tmp0 = tmp0 -> next;
    }
    if (compare(arr1, n0, E))
    {
        n = malloc(sizeof(*n));
        if (!n){
            printf("ERROR: line number %d in function %s\n", __LINE__, __func__);
        }
        n -> element = k;
        n -> next = list;
        list = n;
    }
    return list;
}

bool compare(int *x, struct node *n, int E)
{
    int i;
    while (n != NULL){
        for (i = 0; i < E; i++){
            if (*(x + i) == (*n).element)
                return true;
        }
        n = (*n).next;
    }
    return false;
}

void free_MONOID(struct node **n, int M_MAX)
{
    struct node *tmp1, *tmp2;

    for (int i = 0; i < M_MAX; i++){
        tmp1 = *(n + i);
        while (tmp1 != NULL){
            tmp2 = tmp1 -> next;
            free(tmp1);
            tmp1 = tmp2;
        }
    }
}

void free_dec(struct dec_node **n)
{
    struct dec_node *tmp1, *tmp2;

    for (int i = 0; i < N; i++){
        tmp1 = n[i];
        while (tmp1 != NULL){
            tmp2 = tmp1 -> next;
            free (tmp1);
            tmp1 = tmp2;
        }
    }
}

void free_hash(struct node **n)
{
    struct node *tmp1, *tmp2;

    for (int i = 0; i < N; i++){
        tmp1 = n[i];
        while (tmp1 != NULL){
            tmp2 = tmp1 -> next;
            free (tmp1);
            tmp1 = tmp2;
        }
    }
}

void print_MONOID(struct node **n, int M_MAX)
{
    struct node *tmp0;

    for (int i = 0; i < M_MAX; i++){
        tmp0 = *(n + i);
        while (tmp0 != NULL){
            printf("%i ", tmp0 -> element);
            tmp0 = tmp0 -> next;
        }
        printf("\n");
    }

}

void duplicates(struct dec_node *ptr, int E)
{
    struct dec_node *tmp0, *tmp1, *prev;

    int i, j;
    tmp0 = prev = ptr;
    while (tmp0 -> next != NULL){
        prev = tmp0;
        tmp1 = tmp0 -> next;
        while (tmp1 != NULL){
            //print_nodes("compare", tmp0, tmp1);
            for (j = 0; j < E; j++) {
                 if (tmp0 -> coef[j] != tmp1 -> coef[j]) break;
            }
            if (j == E) {
                /* duplicate */
                prev->next = tmp1->next;
                assert(tmp1 != ptr);
                free(tmp1);
                tmp1 = prev->next;
            }
            else{
                prev = tmp1;
                tmp1 = tmp1->next;
            }
        }
        tmp0 = tmp0 -> next;
    }
}

int size(struct dec_node *ptr)
{
    int counter = 0;
    struct dec_node *tmp = ptr;
    while (tmp != NULL){
        counter++;
        tmp = tmp -> next;
    }
    return counter;
}

int size_hash(struct node *ptr)
{
    int counter = 0;
    struct node *tmp = ptr;
    while (tmp != NULL){
        counter++;
        tmp = tmp -> next;
    }
    return counter;
}

void bubble_sort(int *arr, int n) {
    int i = 0, j = 0, tmp;
    for (i = 0; i < n; i++){   // loop n times - 1 per element
        for (j = 0; j < n - i - 1; j++){ // last i elements are sorted already
            if (arr[j] > arr[j + 1]){  // swop if order is broken
                tmp = arr[j];
                arr[j] = arr[j + 1];
                arr[j + 1] = tmp;
            }
        }
    }
}

// Function to return gcd of a and b
int gcd(int a, int b)
{
    if (a == 0)
        return b;
    return gcd(b % a, a);
}

// Function to find gcd of array of
// numbers
int findGCD(int *arr, int n)
{
    int result = arr[0];
    for (int i = 1; i < n; i++){
        result = gcd(arr[i], result);

        if(result == 1)
            return 1;
    }
    return result;
}
\end{lstlisting}