\chapter{Semigroups and monoids}

\section{Semigroups}

Our study of numerical monoids starts with that of semigroups. A \emph{semigroup} is a pair $(S, +)$ with $S$ a set and $+$ an associative binary operation on $S$. Semigroups do not need inverses nor an identity element like a group does. We will assume all semigroups in this paper are commutative and we will omit the binary operation $+$ and denote the semigroup just by $S$. Thus $\N$ is a semigroup under addition as is the set of positive even integers $2\N$.

A \emph{subsemigroup} is a subset of a semigroup $S$ that is closed under the operation of $S$. Thus, $2\N$ is a subsemigroup of $\N$. 

The intersection of any number of subsemigroups is a subsemigroup. To see this, one must show that the intersection of a set of subsemigroups of $S$ is closed under the operation of $S$. Each element in the intersection is in every subsemigroup and each subsemigroup is closed under the operation of $S$ by definition. Thus the intersection is closed which makes the intersection a subsemigroup.

Let $S$ be a semigroup and let $s_1, \ldots, s_k$ be any collection of elements of $S$. \emph{The smallest subsemigroup that contains $s_1, \ldots, s_k$} is the intersection of all subsemigroups that contain $s_1, \ldots, s_k$. We denote this set $\langle s_1, \ldots, s_k \rangle$ and we have
\[
	\langle s_1, \ldots, s_k \rangle = \left \{\sum_{i = 1}^{k} a_i s_i \mid a_i \in \Z_+  \text{ and at least one } a_i \neq 0 \right \}.
\]
Remember it is not necessary for the additive identity to be an element of a semigroup. 

If $\langle s_1, \ldots, s_k \rangle = S$ for some $k \in \NN$ then we say that $S$ is generated by $s_1, \ldots, s_k$ and $S$ is \emph{finitely generated}. When $s \in \langle s_1, \ldots, s_k \rangle$ is written as $s = \sum_{i = 1}^{k} a_is_i$ with $a_i\in\NN$, then we call this sum a \textit{representation} of $s \in S$.

\section{Monoids}

A \emph{monoid} $M$ is a semigroup with a neutral element $0$. A subset $N$ of a monoid $M$ is a \emph{submonoid} of $M$ if it is a subsemigroup that contains the neutral element. Note that the set $\{0\}$ is a submonoid of $M$ and is called the \emph{trivial submonoid}.

As with the case with semigroups, the intersection of any number of submonoids is a submonoid. Thus, for any subset $N$ of a monoid $M$, \emph{the smallest submonoid containing $N$} is the intersection of all submonoids of $M$ that contain $N$. This is also denoted by $\langle N \rangle$. Whether this is a subsemigroup or a submonoid will be clear from context (just check if it contains $0$). Since $\langle N \rangle$ is closed under the operation of $M$, all elements have the form
\[
	\langle N \rangle = \left \{\sum_{i = 1}^{k} a_i s_i \mid a_i \in \Z_+, \ s_i \in N \right \}.
\]
Notice that
\[
	\langle \emptyset \rangle = \{0\} = \langle 0 \rangle
\]
since $\{0\}$ and the empty set $\emptyset$ are subsets of every submonoid. We will now discuss an important submonoid called a numerical monoid.